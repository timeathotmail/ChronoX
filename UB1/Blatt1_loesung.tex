\documentclass{swp1}
\usepackage[utf8]{inputenc}
\usepackage{amssymb}
\usepackage{url}
\usepackage{amsmath}
\begin{document}

% \maketitle{Nummer}{Abgabedatum}{Tutor-Name}{Gruppennummer}
%           {Teilnehmer 1}{Teilnehmer 2}{Teilnehmer 3}
\maketitle{1}{18.05.2014}{Michaela Bunke}{ChronoX}
          {Tim Ellhoff}{Fadiga	Alpha Falikou}{}
          
\section*{Projektplan}          
\subsection*{Aufgabe 1)}
Bei einer Wohnungsrenovierung sollte man sich zunächst über den Gegenstand der Renovierung Gedanken machen; d.h. festzulegen, welchen Teil der Wohnung die Renovierung betreffen soll. Es macht ja einen großen Unterschied, ob man beispielsweise nur einen bestimmten Bereich der Wohnung, wie das Wohnzimmer, renovieren möchte oder alle Räume der Wohnung. \\
Hat man das dies festgelegt, sollte man grob die Ziele der Renovierung festlegen und aufschreiben, was alles renoviert werden soll (beispielsweise die Auslegeware, Wände, Decken usw.).
Man sollte dann die einzelnen Renovierungsschritte in genaue Phasen bzw. Hauptaktivitäten unterteilen und in eine sinnvolle Reihenfolge bringen.\\
Hat man ein grobes Grundgerüst über die einzelnen Arbeitsschritte, die für die Renovierung nötig sind, zusammengestellt, sollte man dieses zeitlich genauer spezifizieren und einen Meilensteinplan erstellen, in dem für jede Phase des Renovierungsvorhabens genaue zeitliche Daten gefunden werden, die auch entsprechend der Reihenfolge der Arbeiten sinnvoll erscheinen (man sollte z.B. nicht den neuen Teppich verlegen, bevor man tapeziert bzw. gestrichen hat). \\

Dann sollte man sich genau überlegen, was für Ressourcen nötig sind, um die Renovierung erfolgreich durchzuführen. Hier wären in erster Linie die Mitarbeiter und deren Fähigkeiten/Skills zu nennen, die bei der Renovierung mithelfen. Dabei sind auch die Arbeitszeiten der Mitarbeiter mit in die Planung aufzunehmen. \\

Des weiteren muss festgelegt werden, welches Werkzeug und Material man für die Renovierung benötigt. Eng damit verbunden ist der Punkt des Budgets, das für ein solches Renovierungsvorhaben zur Verfügung steht. Hat man beispielsweise ein knappes Budget, sollte man ggf. für den Fußbodenbelag statt Eichenparkett lieber Laminat aus dem Baumarkt wählen. \\
Wichtig hierbei dürfte auch sein, dass man sich rechtzeitig mit der Bestellung bzw. Beschaffung von Materialien (z.B. spezielle Tapeten, Parkettfußboden, Fliesen o.Ä.) kümmert, die ggf. länger auf sich warten lassen.\\

Man sollte nach diesen grundsätzlichen Überlegungen auch klare Verantwortlichkeiten innerhalb der Projektorganisation für die Mitarbeiter festlegen sowie einen Projektleiter bestimmen, der das ganze Vorhaben gewissermaßen leitet und auch direkter Ansprechpartner bei Problemen ist.\\

Eng damit in Verbindung steht ein allgemeines Risiko-/Problemmanagement, das typische Probleme spezifizieren soll, die beim Renovieren auftreten können (Ausfall von Mitarbeitern, fehlerhaftes Arbeiten, Arbeiten verzögern sich, weil bestimmte Arbeiten zuvor nicht erledigt wurden usw.). Für solche Fälle sollen zuvor Problemlösungen definiert werden bzw. Schritte eingeleitet werden, die in solchen Fällen unternommen werden oder zumindest Möglichkeiten zur Schadensbegrenzung aufweisen können.\\

Ebenso sollte man bestimmte Annahmen und Abhängigkeiten genau definieren. Annahmen für erfolgreiches Arbeiten wären z.B. Pünktlichkeit und Verlässlichkeit sowie das Arbeiten nach bestem Wissen und Gewissen der Mitarbeiter oder die Nicht-Möglichkeit zur Verschiebung von Deadlines und festen Terminen.\\
Abhängigkeiten wären z.B., dass man zwingend voraussetzen muss, dass ein Maler sein Werkzeug dabei hat, weil er sonst nicht arbeiten kann; aber auch, dass er ggf. nur eine bestimmte Zeit für das Renovierungsvorhaben am Tag aufwenden kann, also auch noch anderes zu tun hat.\\

Bevor man genaue Arbeitspakete für die Renovierungsschritte erstellt, sollte man sich auf eine Art der Projektüberwachung einigen, um Arbeitsfortschritte, aber auch frühzeitig Probleme zu erkennen. Dies kann man beispielsweise durch tägliche kurze Besprechungen vor Arbeitsbeginn mit dem Renovierungsteam erreichen. \\

Zu guter Letzt sollten die weiter oben spezifizierten Aufgaben und Meilensteine bzw. Phasen nun in Arbeitspaketen genauer beschrieben werden, in denen kurz die Aufgaben zu jeder Arbeitseinheit beschrieben, die menschlichen Ressourcen in Form der Mitarbeiter festgelegt, die Abhängigkeiten zu anderen Arbeiten definiert sowie die einzelnen Arbeitsaufwände abgeschätzt werden. Ebenso muss der Beginn und das Ende der einzelnen Arbeiten in den Arbeitspaketen festlegt werden.\\
Sinnvoll wäre es auch, wenn man diese Arbeitspakete grafisch visualisiert und den Plan für alle Mitarbeiter sichtbar aushängt, sodass jeder den Zeit- und Arbeitsplan im Blick hat.

\subsection*{Aufgabe 2)}

Die Projektplanung sollte grundsätzlich immer so früh wie möglich stattfinden und spätestens dann, wenn man ein Grundgerüst an Zielen und eine grobe Vorstellung davon hat, wie man diese Zielen erreichen kann und das Projekt erfolgreich umsetzen kann. Dies hat mehrere Gründe: je früher die Planung stattfindet, desto
\begin{itemize}
\item  früher kann mit der organisatorischen Arbeit (Mitarbeiter- und Materialzusammenstellung, Budgetfestsetzung usw.) begonnen werden, was bei später Planung fatale Folgen haben könnte (z.B. dann, wenn man bestimmte Mitarbeiter nicht mehr für sich gewinnen kann, weil Terminkonflikte bestehen oder man Material bestellen muss, dass erst geliefert werden würde, wenn man eigentlich schon mit der Renovierung fertig sein wollte usw.)
\item eher können evt. auftretende Probleme erkannt und aus der Welt geschafft werden
\item mehr Zeit bleibt am Ende für eventuelle Nachbesserungen oder Korrekturen bei der Planung
\end{itemize}
\newpage

\section*{Arbeitsplanung}          
\subsection*{Aufgabe 3)}
\subsection*{Aufgabe 4)}
\subsection*{Aufgabe 5)}
\section*{Risiken}     

\subsection*{Aufgabe 6)}
\subsubsection*{R1:}
Wenn der Kunde bei den Vorbesprechungen bzgl. der Anforderungen der zu erstellenden Softwarelösung bereits unstetig ist und den Eindruck vermittelt, dass er selbst nicht so genau weiß, welche Aufgaben die Software bewältigen soll, ist dies ein häufig zu beobachtendes Kundenphänomen. \\
Anders ausgedrückt könnte man auch sagen: Der Kunde weiß selbst nicht, was er will oder braucht, um eine Software zu erhalten, die seinen speziellen Zielen dienlich ist. Dies hat häufig damit zu tun, dass der Kunde wenig bis gar keine Vorstellung davon hat, was eine Software können muss, um bestimmte Aufgaben zu erfüllen oder er kann sich nicht vorstellen, dass Software vielleicht bestimmte Aufgaben auf verschiedenste Weisen bewältigen kann.\\
Eine solche Ausgangssituation führt sehr häufig dazu, dass schon zu Beginn zwischen Kunde und Softwarefirma Missverständnisse entstehen können. Gewissermaßen ist es die Kunst der Softwarefirma, durch Gespräche, Ist-Analysen oder Prototypenvorstelungen beim Kunden herauszufinden, was er wirklich braucht. Dieses Phänomen ist einmal in einem sehr bekannten Cartoon "Was der Kunde wollte" grafisch veranschaulicht worden und bringt es auf den Punkt\footnote{\url{http://www.projectcartoon.com/cartoon/1278}}.\\
Denn wenn der Kunde die Anforderungen, die die Software erfüllen soll, schon falsch oder missverständlich oder widersprüchlich erklärt, führt es häufig dazu, dass die Softwarefirma den Kunden schon falsch versteht (die ersten beiden Bilder im Cartoon). \\
Wenn Anforderungen schon zu Beginn an nicht klar sind wegen der Unstetigkeit des Kunden beliebig interpretierbar sind, führt dies auch häufig dazu, dass die Abteilungen innerhalb der Softwarefirma bereits den Architekturentwurf falsch entwerfen, weil dieser sich aus den falschen Anforderungen unmittelbar ableitet (Bild 3 im Cartoon). \\
Daraus resultiert natürlich auch, dass die Produktion, also in dem Fall die Entwickler in der Implementierungsabteilung, die Software am Kunden vorbei entwickeln (Bild 4) und sich am Ende (Bild 6), wenn es zu spät ist,  herausstellt, was der Kunde wirklich gebraucht hätte. \\

Eintrittswahrscheinlichkeit (auf einer Skala von 1-10, wobei 1: gering, 10: sehr hoch): 7 \\
Schadenshöhe (auf einer Skala von 1-10, wobei 1: gering, 10: sehr hoch): 9

\subsubsection*{R2:}
Wenn die Entwicklergruppe zum ersten Mal zusammenarbeitet, muss das nicht zwangsläufig zu Problemen führen. Dies hängt natürlich genauso stark von den einzelnen Mitgliedern sowie den Hard- und Softskills des Teams ab, aber auch von der Teamszusammensetzung und deren Atmosphäre. \\
Fragen, die man sich hierbei stellen muss und die zu Problemen führen könnten, lauten z.B.: \\
\begin{itemize}
\item Sind die einzelnen Gruppenmitglieder kompetent (genug)?
\item Verstehen sich die Gruppenmitlieder untereinander? Können Sie zusammenarbeiten?
\item Hat das Team Kommunikationsschwierigkeiten?
\item Gibt es nur ''Alphatiere'' oder ggf. Leute, die nicht im Team zusammenarbeiten können?
\end{itemize}
Eines der Hauptprobleme tritt dann ein, wenn sich die Gruppe untereinander nicht versteht oder es Barrieren gibt oder sie nicht mal auf fachlicher Ebene aus verschiedenen Gründen miteinander zielführend kommunizieren können. Dabei spielt auch die Gruppenzusammensetzung eine entscheidende Rolle sowie die Charaktere der einzelnen Mitglieder. Wenn solche Probleme auftreten, führt dies schon sehr häufig über kurz oder lang zum Scheitern des gesamten Projekts, sei es, weil Gruppenmitglieder aussteigen, Deadlines nicht eingehalten werden oder die Gruppe aneinander vorbeiredet und -arbeitet. \\
Des weiteren können sich schwerwiegende Folgen aus der Tatsache, dass die Entwickler sich untereinander noch gar nicht kennen und demzufolge auch nicht einschätzen können, wie der einzelne arbeitet bzw. ob er überhaupt in der Lage ist, bestimmte Teilbereiche zu übernehmen, ergeben. So werden vielleicht Aufgabenbereiche bzw. Arbeitspakete einem Gruppenmitglied in Unkenntnis seiner Fähigkeiten zugeordnet, die unter Umständen nur mangelhaft oder gar nicht bearbeitet werden.\\

Eintrittswahrscheinlichkeit (auf einer Skala von 1-10, wobei 1: gering, 10: sehr hoch): 8 \\
Schadenshöhe (auf einer Skala von 1-10, wobei 1: gering, 10: sehr hoch): 8

\subsubsection*{R3:}
Wenn ein Gruppenmitglied krankheitsbedingt ausfällt, ist dies immer ein großes Problem, auch wenn es in einem guten Projektplan im Bereich  Risikomanagement bereits einkalkuliert werden sollte, da dies immer wieder passieren kann. \\
Folgen, die sich daraus ergeben, sind zunächst einmal, dass bei längerer Krankheit von z.B. mehr als 3-4 Tagen ganze Arbeitspakete und somit Verantwortlichkeiten auf die Schultern von den anderen Mitgliedern des Teams verteilt werden müssen, um den Ausfall zu kompensieren. Es ist auch meist möglich, für eine kurze Zeit Mehrarbeit zu leisten und dennoch seine eigentlichen Aufgaben nicht aus den Augen zu verlieren. \\
Sollte allerdings ein Gruppenmitglied wirklich 4 Wochen ausfallen, führt dies meist unweigerlich dazu, dass die ursprünglich geplanten Arbeiten nicht mehr geschafft werden können und sich somit der ganze Projektplan ändert bzw. Termine verschieben oder Deadlines nicht mehr eingehalten werden können.\\
In einem solchen Fall hat das natürlich Auswirkungen auf das gesamte Projekt und eine Art Dominoeffekt.

Eintrittswahrscheinlichkeit (auf einer Skala von 1-10, wobei 1: gering, 10: sehr hoch): 4 \\
Schadenshöhe (auf einer Skala von 1-10, wobei 1: gering, 10: sehr hoch): 7

\subsubsection*{R4:}
Sollte der Kunde den Einsatz von gleich drei neuen Technologien einfordern, mit denen das Entwicklerteam keinerlei Projekterfahrung haben, führt dies häufig dazu, dass einzelne Mitglieder u.U. unterschiedlich lange benötigen, sich in die neue Materie einzuarbeiten. Dem einen mag dies schnell gelingen und er kann gleich die neue Technologie in seine Arbeit einbinden. Dem anderen gelingt dies entweder nur teilweise (beispielsweise nur zwei der drei Technologien) oder gar nicht. Dieser wird spätestens dann Probleme bekommen, wenn er seine Aufgaben nicht zufriedenstellend oder gar nicht schafft. \\
Das führt dazu, dass Teile des Teams entweder die Aufgaben des in dem Fall inkompetenten Mitglieds übernehmen oder sehr viel Zeit investieren müssen , wodurch auch wieder andere Arbeiten auf der Strecke bleiben, um ihm die neue Technik ggf. zu erläutern oder zu zeigen, wie sie anzuwenden ist.\\

Und selbst wenn alle Gruppenmitglieder sich gleich schnell einarbeiten können, fehlt ja eben auch die Erfahrung mit der Technik. D.h. plötzlich auftretende Probleme oder Ausfälle können u.U. nicht kompensiert werden, was im schlimmsten Fall auch zum Scheitern des gesamten Projekts führen kann, nur, weil man sich ggf. beim Spezifizieren der Anforderungen vom Kunden bzw. von deren Wünschen überfordern lassen hat, ohne abzuschätzen, ob man dem auch gerecht werden kann.

Eintrittswahrscheinlichkeit (auf einer Skala von 1-10, wobei 1: gering, 10: sehr hoch): 3 \\
Schadenshöhe (auf einer Skala von 1-10, wobei 1: gering, 10: sehr hoch): 8

\subsection*{Aufgabe 7)}
\subsubsection*{R1:}
\begin{itemize}
\item Mit dem Kunden frühzeitig Kundengespräche ansetzen und möglichst genaue Fragen zu den Anforderungen der Software stellen, um Missverständnissen vorzubeugen. Außerdem müsste dem nachgegangen werden, warum der Kunde so unstetig und unsicher ist. Die Eintrittswahrscheinlichkeit und Schadenshöhe würden sich dann je um zwei Punkte senken lassen, wenn Unklarheiten oder Unsicherheiten des Kunden bzgl. der Anforderungen bzw. der Aufgaben, die die Software erfüllen soll, ausgeräumt werden könnten.
\item Mittels einer frühen Prototypenerstellung dem Kunden schon einen Einblick in eine mögliche GUI geben, um die Software frühzeitig für den Kunden visuell greifbarer zu machen. Die Eintrittswahrscheinlichkeit und Schadenshöhe könnte sich hier auch um je zwei bis vier Punkte senken lassen, da ein Prototyp ggf. mehr aussagt oder Unklarheiten beseitigt, als bloße Fragen im Kundengespräch.
\item Möglicherweise Einsetzung von Extreme Programming (XP). Dabei erfolgt die Entwicklung nur der Funktionen, die wirklich gebraucht werden. Änderungen sind dabei auch während des Projekts möglich. Der Austausch mit dem Kunden erfolgt ständig und stetig, sodass häufige Meetings dafür sorgen sollen, dass der Kunde z.B. implementierte Anforderungen auch akzeptiert oder Änderungen vornehmen lässt. Die Eintrittswahrscheinlichkeit könnte sich drastisch verringern, wenn die XP-Prinzipien gut umgesetzt werden, ebenso die Schadenshöhe. Allerdings verlangt XP eine extrem hohe Kundenbindung und Dauerverfügbarkeit ab und das wird nicht mit jedem Kunden zu machen sein. In einem solchen Fall könnte die Schadenshöhe sehr hoch schnellen, wenn der Kunde z.B. mitten im Projekt nicht mehr so mit dem Entwicklerteam kommuniziert.

\end{itemize}

\subsubsection*{R2:}
\begin{itemize}
\item Zunächst sollte sich das Team, das sich ja noch nicht kennt, fachlich und menschlich austauschen. Vor allem letzteres ich nicht unwichtig, weil dadurch die Mitglieder sich ein Bild vom anderen machen können und lernen, die anderen einzuschätzen. Regelmäßige Kommunikation uns Treffen auch physischer Art können dazu beitragen. Die Eintrittswahrscheinlichkeit bleibt unverändert, aber die Schadenshöhe kann sich enorm senken lassen, wenn das Team gleich eine gute Atmosphäre und ein positives Arbeitsklima schafft.
\item Sollte es größere Gruppenschwierigkeiten ergeben, die sich nicht so einfach lösen lassen, sollte ein Gruppentreffen einberufen werden und der jeweilige Projektleiter oder Phasenleiter versuchen, Ungereimtheiten oder Streitigkeiten sowie Missverständnisse aus dem Weg zu räumen, um so zu versuchen, das Arbeitsklima wiederherzustellen und sich den eigentlichen Aufgaben widmen zu können. Die Eintrittswahrscheinlichkeit sowie die Schadenshöhe würden sich dann schnell senken lassen, wenn auf diese Weise Konflikte oder Barrieren aus dem Weg geräumt werden könnten.
\item
\end{itemize}
\subsubsection*{R3:}
\begin{itemize}
\item 
\item
\item
\end{itemize}
\subsubsection*{R4:}
\begin{itemize}
\item 
\item
\item
\end{itemize}

\subsection*{Aufgabe 8)}


\end{document}

