\documentclass{swp1}
\usepackage[utf8]{inputenc}
\usepackage{amssymb}
\begin{document}

% \maketitle{Nummer}{Abgabedatum}{Tutor-Name}{Gruppennummer}
%           {Teilnehmer 1}{Teilnehmer 2}{Teilnehmer 3}
\maketitle{1}{18.05.2014}{Michaela Bunke}{ChronoX}
          {Tim Ellhoff}{Fadiga Alpha Falikou}{}

\section*{Projektplan}          
\subsection*{Aufgabe 1)}
Bei einer Wohnungsrenovierung sollte man sich zunächst über den Gegenstand der Renovierung Gedanken machen; d.h. festzulegen, welchen Teil der Wohnung die Renovierung betreffen soll. Es macht ja einen großen Unterschied, ob man beispielsweise nur einen bestimmten Bereich der Wohnung, wie das Wohnzimmer, renovieren möchte oder alle Räume der Wohnung. \\
Hat man das dies festgelegt, sollte man grob die Ziele der Renovierung festlegen und aufschreiben, was alles renoviert werden soll (beispielsweise die Auslegeware, Wände, Decken usw.).
Man sollte dann die einzelnen Renovierungsschritte in genaue Phasen bzw. Hauptaktivitäten unterteilen und in eine sinnvolle Reihenfolge bringen.\\
Hat man ein grobes Grundgerüst über die einzelnen Arbeitsschritte, die für die Renovierung nötig sind, zusammengestellt, sollte man dieses zeitlich genauer spezifizieren und einen Meilensteinplan erstellen, in dem für jede Phase des Renovierungsvorhabens genaue zeitliche Daten gefunden werden, die auch entsprechend der Reihenfolge der Arbeiten sinnvoll erscheinen (man sollte z.B. nicht den neuen Teppich verlegen, bevor man tapeziert bzw. gestrichen hat). \\

Dann sollte man sich genau überlegen, was für Ressourcen nötig sind, um die Renovierung erfolgreich durchzuführen. Hier wären in erster Linie die Mitarbeiter und deren Fähigkeiten/Skills zu nennen, die bei der Renovierung mithelfen. Dabei sind auch die Arbeitszeiten der Mitarbeiter mit in die Planung aufzunehmen. \\

Des weiteren muss festgelegt werden, welches Werkzeug und Material man für die Renovierung benötigt. Eng damit verbunden ist der Punkt des Budgets, das für ein solches Renovierungsvorhaben zur Verfügung steht. Hat man beispielsweise ein knappes Budget, sollte man ggf. für den Fußbodenbelag statt Eichenparkett lieber Laminat aus dem Baumarkt wählen. \\
Wichtig hierbei dürfte auch sein, dass man sich rechtzeitig mit der Bestellung bzw. Beschaffung von Materialien (z.B. spezielle Tapeten, Parkettfußboden, Fliesen o.Ä.) kümmert, die ggf. länger auf sich warten lassen.\\

Man sollte nach diesen grundsätzlichen Überlegungen auch klare Verantwortlichkeiten innerhalb der Projektorganisation für die Mitarbeiter festlegen sowie einen Projektleiter bestimmen, der das ganze Vorhaben gewissermaßen leitet und auch direkter Ansprechpartner bei Problemen ist.\\

Eng damit in Verbindung steht ein allgemeines Risiko-/Problemmanagement, das typische Probleme spezifizieren soll, die beim Renovieren auftreten können (Ausfall von Mitarbeitern, fehlerhaftes Arbeiten, Arbeiten verzögern sich, weil bestimmte Arbeiten zuvor nicht erledigt wurden usw.). Für solche Fälle sollen zuvor Problemlösungen definiert werden bzw. Schritte eingeleitet werden, die in solchen Fällen unternommen werden oder zumindest Möglichkeiten zur Schadensbegrenzung aufweisen können.\\

Ebenso sollte man bestimmte Annahmen und Abhängigkeiten genau definieren. Annahmen für erfolgreiches Arbeiten wären z.B. Pünktlichkeit und Verlässlichkeit sowie das Arbeiten nach bestem Wissen und Gewissen der Mitarbeiter oder die Nicht-Möglichkeit zur Verschiebung von Deadlines und festen Terminen.\\
Abhängigkeiten wären z.B., dass man zwingend voraussetzen muss, dass ein Maler sein Werkzeug dabei hat, weil er sonst nicht arbeiten kann; aber auch, dass er ggf. nur eine bestimmte Zeit für das Renovierungsvorhaben am Tag aufwenden kann, also auch noch anderes zu tun hat.\\

Bevor man genaue Arbeitspakete für die Renovierungsschritte erstellt, sollte man sich auf eine Art der Projektüberwachung einigen, um Arbeitsfortschritte, aber auch frühzeitig Probleme zu erkennen. Dies kann man beispielsweise durch tägliche kurze Besprechungen vor Arbeitsbeginn mit dem Renovierungsteam erreichen. \\

Zu guter Letzt sollten die weiter oben spezifizierten Aufgaben und Meilensteine bzw. Phasen nun in Arbeitspaketen genauer beschrieben werden, in denen kurz die Aufgaben zu jeder Arbeitseinheit beschrieben, die menschlichen Ressourcen in Form der Mitarbeiter festgelegt, die Abhängigkeiten zu anderen Arbeiten definiert sowie die einzelnen Arbeitsaufwände abgeschätzt werden. Ebenso muss der Beginn und das Ende der einzelnen Arbeiten in den Arbeitspaketen festlegt werden.\\
Sinnvoll wäre es auch, wenn man diese Arbeitspakete grafisch visualisiert und den Plan für alle Mitarbeiter sichtbar aushängt, sodass jeder den Zeit- und Arbeitsplan im Blick hat.

\subsection*{Aufgabe 2)}

Die Projektplanung sollte grundsätzlich immer so früh wie möglich stattfinden und spätestens dann, wenn man ein Grundgerüst an Zielen und eine grobe Vorstellung davon hat, wie man diese Zielen erreichen kann und das Projekt erfolgreich umsetzen kann. Dies hat mehrere Gründe: je früher die Planung stattfindet, desto
\begin{itemize}
\item  früher kann mit der organisatorischen Arbeit (Mitarbeiter- und Materialzusammenstellung, Budgetfestsetzung usw.) begonnen werden, was bei später Planung fatale Folgen haben könnte (z.B. dann, wenn man bestimmte Mitarbeiter nicht mehr für sich gewinnen kann, weil Terminkonflikte bestehen oder man Material bestellen muss, dass erst geliefert werden würde, wenn man eigentlich schon mit der Renovierung fertig sein wollte usw.)
\item eher können evt. auftretende Probleme erkannt und aus der Welt geschafft werden
\item mehr Zeit bleibt am Ende für eventuelle Nachbesserungen oder Korrekturen bei der Planung
\end{itemize}
\newpage

\section*{Arbeitsplanung}          
\subsection*{Aufgabe 3)}
\subsection*{Aufgabe 4)}
\subsection*{Aufgabe 5)}
\section*{Risiken}     
\subsection*{Aufgabe 6)}
\subsection*{Aufgabe 7)}
\subsection*{Aufgabe 8)}
     







\end{document}

