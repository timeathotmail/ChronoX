\documentclass{swp1}
\usepackage[utf8]{inputenc}
\usepackage{amssymb}
\usepackage{url}


% Tabellen
\usepackage{tabularx}
\usepackage{supertabular}
\usepackage{booktabs}
\usepackage{listings}






\begin{document}

% \maketitle{Nummer}{Abgabedatum}{Tutor-Name}{Gruppennummer}
%           {Teilnehmer 1}{Teilnehmer 2}{Teilnehmer 3}
\maketitle{5}{27.07.2014}{Michaela Bunke}{ChronoX}
          {Tim Ellhoff}{Karsten Betjemann}{}
          
\section*{Aufgabe 1)}

\begin{lstlisting}
public abstract class Command{
	
	protected char changedChar; // Wird in den Klassen Edit UND Del benötigt
	
	public abstract void execute();
	public abstract void undo();
	public abstract void redo();

}

public class Edit extends Command{
	
	public Edit(char c){
		changedChar = c;
	}
	
	public void execute(){
		TextBuffer.add(changedChar);
	}
	
	public void undo(){
		TextBuffer.del();
	}
	
	public void redo(){
		execute();
	}

}

public class Del extends Command{
	
	public void execute(){
		changedChar = TextBuffer.del();
	}
	
	public void undo(){
		TextBuffer.add(changedChar);
	}
	
	public void redo(){
		execute();
	}

}
\end{lstlisting}


\section*{Aufgabe 2) und Aufgabe 3)}

\begin{lstlisting}
public class InvalidArgumentException extends Throwable{};

public abstract class Medium{
	
	private String titel;
	private String verlag;
	// --- Aufgabe 3 (section-start)
	protected double rating;
	protected int ratings;
	// --- Aufgabe 3 (section-end)
	
	public void setTitel(String titel){
		if ( titel!=null & !titel.equals("") ){
			this.titel = titel;
		} else {
			throw new InvalidArgumentException();
		}
	}

	public void setVerlag(String verlag){
		if ( verlag!=null & !verlag.equals("") ){
			this.verlag = verlag;
		} else {
			throw new InvalidArgumentException();
		}
	}
	
	public String getTitel(){
		return titel;
	}
	
	public String getVerlag(){
		return verlag;
	}
	
	// --- Aufgabe 3 (section-start)
	public abstract void setRating( int rating );
	public abstract double getRating();
	// --- Aufgabe 3 (section-end)
	
}


public class Buch extends Medium{
	
	private List<String> authors;
	private Buchserie bookSet;
	
	public Buch( String titel, String verlag, String author ){
		setTitel(titel);
		setVerlag(verlag);
		if ( author!=null & !author.equals("") ){
			this.authors.add(author);
		} else {
			throw new InvalidArgumentException();
		}
	}
	
	public List<String> getAuthors(){
		return authors;
	}
	
	public addAuthor( String author ){
		if ( !authors.contains(author) ){
			authors.add(author);
		}
	}
	
	public delAuthor( String author ){
		if ( authors.contains(author) ){
			authors.remove(author);
		}
	}
	
	public getBookSet(){
		return bookSet;
	}
	
	public setBookSet( Buchserie bs ){
		if ( !bs.getBooks.contains(this) ){
			bs.getBooks.add(this);
		}
		bookSet = bs;
	}
	
	// --- Aufgabe 3 (section-start)
	public void setRating( int rating ){
		if ( rating>=1 & rating <=5 ){
			this.rating = (this.rating*ratings+rating)/(++ratings);
		} else {
			throw new InvalidArgumentException();
		}
	}
	
	public double getRating(){
		return rating;
	}
	// --- Aufgabe 3 (section-end)
	
}

public class Buchserie extends Medium{
	
	private List<Buch> books;
	
	public Buchserie( String titel, String verlag, Buch firstBook ){
		setTitel(titel);
		setVerlag(verlag);
		if ( firstBook!=null ){
			this.books.add(firstBook);
		} else {
			throw new InvalidArgumentException();
		}
	}
	
	public List<Buch> getBooks(){
		return books;
	}
	
	// --- Aufgabe 3 (section-start)
	public void setRating( int rating ){
		if ( rating>=1 & rating <=5 ){
			this.rating = (this.rating*ratings+rating)/(++ratings);
		} else {
			throw new InvalidArgumentException();
		}
	}
	
	public double getRating(){
		double averageRating = 0;
		
		for ( Buch b: books ){
			averageRating += b.getRating();
		}
		averageRating = averageRating / books.size();
		
		return averageRating;
	}
	// --- Aufgabe 3 (section-end)
	
}
\end{lstlisting}

\begin{figure}[h]
\centering{\includegraphics[width=16.5cm]{uml_klassendiAgramm.png}}
\caption{UML Klassendiagramm zum Sachverhalt in Aufgabe 2}
\label{ab3}
\end{figure}

\section*{Aufgabe 4)}
\section*{Aufgabe 5)}

\end{document}

